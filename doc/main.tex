
\documentclass{article}

\usepackage[toc,page]{appendix}
\usepackage{hyperref}
\usepackage[margin=3cm]{geometry}

\newcommand{\bril}{\href{https://capra.cs.cornell.edu/bril/}{bril}}
\newcommand{\Bril}{\href{https://capra.cs.cornell.edu/bril/}{Bril}}
\newcommand{\refint}{\href{https://capra.cs.cornell.edu/bril/tools/interp.html}
  {reference interpreter}}

\title{BRB Specification \\
\Large{Big Red Bytecode}}
\author{Susan Garry, Charles Sherk}


\begin{document}
\maketitle
\section{Overview}
We provide a description of a bytecode for \bril{}. We support the standard
extensions, and make it relatively easy to extend the bytecode to support new
extensions as well. We also provide a reference for the opcodes that we have
chosen in appendix \ref{app:opcodes}. The general encoding is that each
instruction is 64 bits, but some instructions require more information. These
instructions encode in some multiple of 64 bits which can be determined by
reading the first 64 bits.
\section{Base Instructions}
Temporaries (stack variables) are represented by a number in the range
$[0, 2^{16})$. This does technically impose a limit over the original \bril{}
implementation, but we don't believe that it will be relevant for any
programs. Since all data types take up 8 bytes in our implementation, this comes
out to half a megabyte of space maximum for each stack frame. In java, the
maximum stack size is 1 megabyte, so this seems like a reasonable limitation.

% TODO alternate encoding for this?
We also limit the number of instructions to $2^{16}$ per function so that labels
can be represented in 16 bits as instruction indices. This allows us to use the
following general instruction format:\\
\begin{center}
  \begin{tabular}{r|c|c|c|c|c|}
    \cline{2-6}
    & \texttt{labelled} & \texttt{opcode} & \texttt{dest} & \texttt{arg1}
    & \texttt{arg2} \\ \cline{2-6}
    Size (bits): & 1 & 15 & 16 & 16 & 16 \\ \cline{2-6}
  \end{tabular}
\end{center}
The \texttt{labelled} bit marks if an instruction has a label immediately preceding
it, that is, it can be a jump target. This is important for faithful execution
of the $\phi$ instruction, as it needs to keep track of the most recent label.

The \texttt{opcode} is 15 bits, which is far bigger than we need, but it makes
sense to keep everything in 16 bit increments. There is space for another couple
bits of tagging extensions after the label bit since we don't use many opcode
bits, but there is no need for them yet.

Next is the \texttt{dest} temporary. This is where the result of this
instruction will be put. \texttt{arg1} and \texttt{arg2} are the two
arguments. This works for all instructions of the form
\[
  \texttt{dest} := \texttt{arg1}\ op\ \texttt{arg2}
\]
There are however other formats.
\section{Branch instructions}
Branches are the simplest instructions that don't fit into the ``base'' format,
and are laid out as follows:\\
\begin{center}
  \begin{tabular}{r|c|c|c|c|c|}
    \cline{2-6}
    & \texttt{labelled} & \texttt{opcode} & \texttt{test} & \texttt{ltrue}
    & \texttt{lfalse} \\ \cline{2-6}
    Size (bits): & 1 & 15 & 16 & 16 & 16 \\ \cline{2-6}
  \end{tabular}
\end{center}
The \texttt{labelled} and \texttt{opcode} sections work the same way as base
instructions (the opcode will always be the branch opcode, but we still need it
to be able to decode easily). \texttt{test} is a temporary which if it is 1 will
transfer control flow to \texttt{ltrue}, and if it is 0 will transfer to
\texttt{lfalse}. Other values result in undefined behavior, as we only allow
booleans to be 0 or 1.
\section{Constant Instruction}
Constants present a bit of an issue: \bril{} supports constants up to 64 bits,
but that would be the whole instruction. We also observe that usually constants
are much smaller, so we provide two encodings. First of all is the ``standard''
encoding:\\ 
\begin{center}
  \begin{tabular}{r|c|c|c|c|}
    \cline{2-5}
    & \texttt{labelled} & \texttt{opcode} & \texttt{dest} & \texttt{value}\\
    \cline{2-5}
    Size (bits): & 1 & 15 & 16 & 32 \\ \cline{2-5}
  \end{tabular}
\end{center}
This works for integers which can be represented in 32 bits as well as booleans,
but for larger integers and floats, there is a loss of precision. We also
provide a ``long constant'' format as follows:
\begin{center}
  \begin{tabular}{r|c|c|c|c|c|c|}
    \cline{2-7}
    & \texttt{labelled} & \texttt{opcode} & \texttt{dest} & \texttt{type}
    & \texttt{unused}
    & \texttt{value} \\ \cline{2-7}
    Size (bits): & 1 & 15 & 16 & 16 & 48 & 64 \\ \cline{2-5}
  \end{tabular}
\end{center}
This is 128 bits instead of 64, as we use \texttt{labelled}, \texttt{opcode},
and \texttt{dest}
the same way.
We also include the type since we have the space, and this is necessary to
translate back to \bril{}.
The next 64 bits are the constant. This opcode is distinct
from the opcode for the ``standard'' constant instruction encoding, so it allows
distinction.
\section{Function Call Instruction}
\Bril{} supports function calls in a single instruction with an unlimited number
of arguments, which is obviously a problem for a 64 bit encoding. We provide an
initial encoding as follows:
\begin{center}
  \begin{tabular}{r|c|c|c|c|c|}
    \cline{2-6}
    & \texttt{labelled} & \texttt{opcode} & \texttt{dest} & \texttt{num\_args}
    & \texttt{target} \\ \cline{2-6}
    Size (bits): & 1 & 15 & 16 & 16 & 16 \\ \cline{2-6}
  \end{tabular}
\end{center}
\texttt{labelled}, \texttt{opcode}, and \texttt{dest} work as in the standard
encoding. \texttt{num\_args} tells us the number of arguments to the function, so
we know how many of the following 64 bit chunks contain them, and
\texttt{target} is the function to call. Functions are represented as their
indices in order they appear in the source, which makes linking multiple sources
together impossible at the bytecode stage. For this reason, the outputted file
includes the name of the function, so that it is possible to reparse and link
multiple files.

We then have a sequence of 64 bit words that are split into 4 arguments, not all
of which are used. These are represented as temps.

\section{Print Instruction}
\Bril{} also includes a fairly high level \texttt{print} instruction, which
deals with it's arguments differently depending on their types. In order to
encode this, we need to have an encoding for the types supported by \bril{},
which we provide in appendix \ref{app:types}. The first word of a \texttt{print}
is encoded as follows:
\begin{center}
  \begin{tabular}{r|c|c|c|c|c|}
    \cline{2-6}
    & \texttt{labelled} & \texttt{opcode} & \texttt{num\_prints} & \texttt{type1}
    & \texttt{arg1} \\ \cline{2-6}
    Size (bits): & 1 & 15 & 16 & 16 & 16 \\ \cline{2-6}
  \end{tabular}
\end{center}
As usual, \texttt{labelled} and \texttt{opcode} work as
above. \texttt{num\_prints} is the number of arguments to the \texttt{print}
instruction, as \bril{} supports an arbitrary amount. \texttt{type1} is the type
of the first argument, and \texttt{arg1} is the encoding of the
temporary. Subsequent arguments come in a sequence of 64 bit chunks which hold
up to two arguments each, alternating the type and then the temp.
\section{$\Phi$ Instruction}
In order to properly mimic the behavior of the \refint, we need to be able to
faithfully execute the $\Phi$ instruction on SSA programs. This is the reason we
have the bit to keep track of which instructions were labelled in the
source. The $\Phi$ instruction also supports an arbitrary number of arguments,
so we use multiple 64 bit words to encode it. The first is as follows:
\begin{center}
  \begin{tabular}{r|c|c|c|c|c|}
    \cline{2-6}
    & \texttt{labelled} & \texttt{opcode} & \texttt{dest} & \texttt{num\_choices}
    & \texttt{unused} \\ \cline{2-6}
    Size (bits): & 1 & 15 & 16 & 16 & 16 \\ \cline{2-6}
  \end{tabular}
\end{center}
\texttt{labelled}, \texttt{opcode}, and \texttt{dest} work the same as a standard
instruction, and \texttt{num\_choices} is the number of places we might come
from into the $\Phi$ instruction. Following this instruction are a sufficient
number of 64 bit words to hold all the choices, which are represented as the
encoding of a label followed by the value, which is a temp. Remember labels are
indices into the instruction list.
\begin{appendices}
\section{Opcodes}
\label{app:opcodes}
\begin{tabular}{cccc}
  \begin{minipage}{.25\linewidth}
    \begin{tabular}{|r|l|}\hline
\multicolumn{2}{|c|}{base Instructions}\\ \hline
CONST & 1 \\ \hline
ADD & 2 \\ \hline
MUL & 3 \\ \hline
SUB & 4 \\ \hline
DIV & 5 \\ \hline
EQ & 6 \\ \hline
LT & 7 \\ \hline
GT & 8 \\ \hline
LE & 9 \\ \hline
GE & 10 \\ \hline
NOT & 11 \\ \hline
AND & 12 \\ \hline
OR & 13 \\ \hline
JMP & 14 \\ \hline
BR & 15 \\ \hline
CALL & 16 \\ \hline
RET & 17 \\ \hline
PRINT & 18 \\ \hline
LCONST & 19 \\ \hline
NOP & 20 \\ \hline
ID & 21 \\ \hline
\end{tabular}

  \end{minipage}
  &
    \begin{minipage}{.25\linewidth}
      \begin{tabular}{|r|l|}\hline
\multicolumn{2}{|c|}{ssa Instructions}\\ \hline
PHI & 22 \\ \hline
\end{tabular}

    \end{minipage}
  &
    \begin{minipage}{.25\linewidth}
      \begin{tabular}{|r|l|}\hline
\multicolumn{2}{|c|}{mem Instructions}\\ \hline
ALLOC & 23 \\ \hline
FREE & 24 \\ \hline
STORE & 25 \\ \hline
LOAD & 26 \\ \hline
PTRADD & 27 \\ \hline
\end{tabular}

    \end{minipage}
  &
    \begin{minipage}{.25\linewidth}
      \begin{tabular}{|r|l|}\hline
\multicolumn{2}{|c|}{float Instructions}\\ \hline
FADD & 28 \\ \hline
FMUL & 29 \\ \hline
FSUB & 30 \\ \hline
FDIV & 31 \\ \hline
FEQ & 32 \\ \hline
FLT & 33 \\ \hline
FLE & 34 \\ \hline
FGT & 35 \\ \hline
FGE & 36 \\ \hline
\end{tabular}

    \end{minipage}
\end{tabular}
  \begin{minipage}{\linewidth}
    \section{Types}
    \label{app:types}
    \begin{center}
      \begin{tabular}[H!]{|r|l|}\hline
\multicolumn{2}{|c|}{Types}\\ \hline
BRILINT & 0 \\ \hline
BRILBOOL & 1 \\ \hline
BRILFLOAT & 2 \\ \hline
BRILPTR & 3 \\ \hline
\end{tabular}

    \end{center}
  \end{minipage}
\end{appendices}
\end{document}
